%% Lecture 1

\documentclass{beamer}\usepackage{graphicx, color}
%% maxwidth is the original width if it is less than linewidth
%% otherwise use linewidth (to make sure the graphics do not exceed the margin)
\makeatletter
\def\maxwidth{ %
  \ifdim\Gin@nat@width>\linewidth
    \linewidth
  \else
    \Gin@nat@width
  \fi
}
\makeatother

\IfFileExists{upquote.sty}{\usepackage{upquote}}{}
\definecolor{fgcolor}{rgb}{0.2, 0.2, 0.2}
\newcommand{\hlnumber}[1]{\textcolor[rgb]{0,0,0}{#1}}%
\newcommand{\hlfunctioncall}[1]{\textcolor[rgb]{0.501960784313725,0,0.329411764705882}{\textbf{#1}}}%
\newcommand{\hlstring}[1]{\textcolor[rgb]{0.6,0.6,1}{#1}}%
\newcommand{\hlkeyword}[1]{\textcolor[rgb]{0,0,0}{\textbf{#1}}}%
\newcommand{\hlargument}[1]{\textcolor[rgb]{0.690196078431373,0.250980392156863,0.0196078431372549}{#1}}%
\newcommand{\hlcomment}[1]{\textcolor[rgb]{0.180392156862745,0.6,0.341176470588235}{#1}}%
\newcommand{\hlroxygencomment}[1]{\textcolor[rgb]{0.43921568627451,0.47843137254902,0.701960784313725}{#1}}%
\newcommand{\hlformalargs}[1]{\textcolor[rgb]{0.690196078431373,0.250980392156863,0.0196078431372549}{#1}}%
\newcommand{\hleqformalargs}[1]{\textcolor[rgb]{0.690196078431373,0.250980392156863,0.0196078431372549}{#1}}%
\newcommand{\hlassignement}[1]{\textcolor[rgb]{0,0,0}{\textbf{#1}}}%
\newcommand{\hlpackage}[1]{\textcolor[rgb]{0.588235294117647,0.709803921568627,0.145098039215686}{#1}}%
\newcommand{\hlslot}[1]{\textit{#1}}%
\newcommand{\hlsymbol}[1]{\textcolor[rgb]{0,0,0}{#1}}%
\newcommand{\hlprompt}[1]{\textcolor[rgb]{0.2,0.2,0.2}{#1}}%

\usepackage{framed}
\makeatletter
\newenvironment{kframe}{%
 \def\at@end@of@kframe{}%
 \ifinner\ifhmode%
  \def\at@end@of@kframe{\end{minipage}}%
  \begin{minipage}{\columnwidth}%
 \fi\fi%
 \def\FrameCommand##1{\hskip\@totalleftmargin \hskip-\fboxsep
 \colorbox{shadecolor}{##1}\hskip-\fboxsep
     % There is no \\@totalrightmargin, so:
     \hskip-\linewidth \hskip-\@totalleftmargin \hskip\columnwidth}%
 \MakeFramed {\advance\hsize-\width
   \@totalleftmargin\z@ \linewidth\hsize
   \@setminipage}}%
 {\par\unskip\endMakeFramed%
 \at@end@of@kframe}
\makeatother

\definecolor{shadecolor}{rgb}{.97, .97, .97}
\definecolor{messagecolor}{rgb}{0, 0, 0}
\definecolor{warningcolor}{rgb}{1, 0, 1}
\definecolor{errorcolor}{rgb}{1, 0, 0}
\newenvironment{knitrout}{}{} % an empty environment to be redefined in TeX

\usepackage{alltt}
\usetheme{Stats}
\setbeamercovered{transparent}
\usepackage{color}
\usepackage{hyperref}
  \hypersetup{
		colorlinks=true
		linkcolor=black
		}
\usepackage{url}
\usepackage{graphics}
\usepackage{tikz}
\usepackage{booktabs}
\usepackage{CJKutf8}

\newenvironment{Korean}{%
  \CJKfamily{mj}}{}

%%%%%%%%%%%%%%%%%%%%%%%%%%%%%%%% Title Slide %%%%%%%%%%%%%%%%%%%%%%%%%%
\title[Lecture 1:]{Intro to Social Science Data Analysis \\[1cm] Lecture 1: \\[0.25cm] Introduction to the Course \& R}
\author[]{
    \href{mailto:gandrud@yonsei.ac.kr}{Christopher Gandrud}
}
\date{\today}


\begin{document}

\begin{CJK}{UTF8}{}


\frame{\titlepage}

\section[Outline]{}
\frame{\tableofcontents}

%%%%%%%%%%%%%%%%%%%%%%%%%%%%%%%%% Contact %%%%%%%%%%%%%%%%%%%%%%%%%%%%%%%%%%%%%
\section{Contact}
\frame{
    \frametitle{Contact}
    {\LARGE{Christopher Gandrud}} \\[0.5cm]
        \begin{itemize}
            \item<1-> {\bf{Email:}} \href{mailto:gandrud@yonsei.ac.kr}{gandrud@yonsei.ac.kr}
            \item<2-> {\bf{Office Hours:}} 15:00-17:00 Wednesday \begin{Korean} (정208) \end{Korean}
            \item<3-> {\bf{Open Door:}} You can always come to my office or email me.
        \end{itemize}
}

\frame{
  \frametitle{What should you call me?}
    \begin{center}
      {\LARGE{Please call me Christopher or Chris \\[0.5cm]
      
      In this course we respect {\bf{knowledge}} \& {\bf{evidence}}, not titles.      
      }}
    \end{center}
}

\frame{
  \frametitle{Question}
    \begin{center}
      {\LARGE{What experience do you have reading and using data analysis? \\[0.75cm]
      What do you want to be able to do?}}
    \end{center}
}

\section{Course Aims}

\frame{
  \frametitle{Course Aims (1)}
    {\LARGE{Why take this course?}} \\[0.5cm]
      \begin{itemize}
        \item<1-> Learn how to take {\bf{raw data}}, {\bf{explore it}}, and {\bf{present what you find}}.
        \item<2-> This course {\bf{is hands on}} and {\bf{practical}}. (If you want a mathmatical introduction to statistics, I recommend taking a coure in the Statistics department.)
        \item<3-> The skills taught in this course are important for many {\bf{real-world}} situations: \vspace{0.5cm}
          \begin{enumerate}
            \item<4-> Academics
            \item<5-> Government
            \item<6-> Buisiness
            \item<7-> Journalism
          \end{enumerate}
      \end{itemize}
}

\frame{
  \frametitle{Course Aims (2)}
  {\LARGE{Specifics}} \\[0.5cm]
    \begin{itemize}
      \item<1-> We will learn how do these things with the {\bf{R}} statistical language in the program RStudio.
        \begin{itemize}
      \item<2-> R is difficult when you start to use it, but it is very {\bf{powerfull}} and being able to use it is a very {\bf{marketable skill}}.
        \end{itemize}
      \item<3-> Finally, the course is {\bf{not}} about memorisation. 
      \item<4-> It is about developing tools to {\bf{solve new and unexpected problems}}. 
    \end{itemize}
}

\frame{
  \frametitle{Course Aims (3)}
  \begin{center}
    {\LARGE{This course is useful.}}
  \end{center}
}

\section{Prerequisites}
\frame{
  \frametitle{Prerequisites}
  {\LARGE{This course is intended for beginners.}} \\[0.5cm]
    \begin{itemize}
      \item<1-> You should have good {\bf{basic computer skills}} (Have used Microsoft Excel, for example.)
      \item<2-> You need to be {\bf{curious}}: Why did that happen? How can I solve this problem?
      \item<3-> You need to be {\bf{patient}}: Can't give up why you don't succeed the first time.
    \end{itemize}
}

\section{Course Outline}
\frame{
  \frametitle{Course Outline (1)}
  {\LARGE{Part I: Introduction to Data Gathering \& Management}} \\[0.5cm]
    \begin{itemize}
      \item<1-> R Basics: installing, objects, assignment, functions
      \item<2-> Data structures
      \item<3-> Gathering data
      \item<4-> Replication!
    \end{itemize}
}

\frame{
  \frametitle{Course Outline (2)}
  {\LARGE{Part II: Basic Data Analysis \& Visualisation}} \\[0.5cm]
    \begin{itemize}
      \item<1-> Descriptive statistics
      \item<2-> Data visualisation
      \item<3-> Overview of statistical inference
      \item<4-> Statistical inference with large samples
    \end{itemize}
}

\frame{
  \frametitle{Course Outline (3)}
  {\LARGE{Part III: Introduction to Linear Regression}} \\[0.5cm]
    \begin{itemize}
      \item<1-> Simple linear regression
      \item<2-> Multiple linear regression
      \item<3-> Presenting regression results
    \end{itemize}
}

\frame{
  \frametitle{Course Outline (4)}
  {\LARGE{Part IV: Introduction to Linear Regression}} \\[0.5cm]
    \begin{center}
      {\LARGE{Research Projects: \\[0.5cm] Use all of these skills.}}
    \end{center}
}

\section{Course Materials}

\frame{
  \frametitle{Course Materials (1)}
  {\LARGE{Blog}}: \url{http://yonsei-data-analysis.tumblr.com/} \\[0.25cm]
  {\bf{Password Protected:}} YonseiData \\[0.5cm]
  
  {\LARGE{Syllabus}}: \url{http://bit.ly/QwE4UM}
}

\frame{
  \frametitle{Course Materials (2)}
  
  {\LARGE{Reading}} \\[0.5cm]
  
  The main {\bf{text}} is: {\emph{OpenIntro Statistics (first edition)}} \\
  It is {\bf{free}} and can be downloaded here: \url{http://www.openintro.org/stat/downloads.php}. \\[0.5cm]
  
  You might also want to get: Crawley, Michael J. 2005. {\emph{Statistics: An Introduction Using R}}. Chichester: John Wiley & Sons. Ltd.
}

\frame{
  \frametitle{Course Materials (3)}
  However, the course is more about {\bf{doing}} than {\bf{consuming}}. \\
  
  So the focus is on {\bf{completing tasks}}, not reading. \\[0.5cm]
  
  To help you complete tasks, we are building a course {\bf{Wiki}}: \url{http://bit.ly/NkdgfW}.
}

\section{Assessment}
\frame{
  \frametitle{Assessment (1)}
    \begin{itemize}
      \item<1-> 10\% Class Attendance and Participation
      \item<2-> 40\% 5 Short Assignments: Due weeks 3, 5, 7, 9, 11
      \item<3-> 50\% Pair Research project (paper and presentation): Due Week 16
    \end{itemize}
}

\frame{
  \frametitle{Assessment (2)}
    {\LARGE{Attendance and Participation}} \\[0.5cm]
    
    You must {\bf{attend}} all lectures and seminars.\\[0.25cm]
    
    You must {\bf{participate}} in class discussions and activities.
    
    \begin{itemize}
      \item<2-> Each $1 + N$ absence = $-5$ participation points.
    \end{itemize}    
}

\frame{
  \frametitle{Assessment (3)}
  \begin{center}
    {\LARGE{More details will be given in future classes about the Short Assignments \& Research Project}}
  \end{center}
}

\frame{
  \frametitle{Now}
    \begin{enumerate}
      \item<1-> {\large{Look at the course blog}} (\url{http://yonsei-data-analysis.tumblr.com/})\\[0.25cm]
        \begin{center}
          {\bf{Please post!}} \\[0.5cm]
        \end{center}
  
      \item<2-> {\large{Install R \& RStudio if you want to use your own computer. (RECOMMENDED)}} \\[0.25cm]
      You can find instructions on the course wiki page: \url{http://bit.ly/PBjDdw}\\[0.5cm]
  
      \item <3->{\large{Open and play around with RStudio}}
      \item <4->{\large{Get a Dropbox account (\url{https://www.dropbox.com/})}}
    \end{enumerate}
}


\end{CJK}
\end{document}
