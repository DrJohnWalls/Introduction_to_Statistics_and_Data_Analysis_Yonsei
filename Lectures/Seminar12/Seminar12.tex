\documentclass{beamer}\usepackage{graphicx, color}
%% maxwidth is the original width if it is less than linewidth
%% otherwise use linewidth (to make sure the graphics do not exceed the margin)
\makeatletter
\def\maxwidth{ %
  \ifdim\Gin@nat@width>\linewidth
    \linewidth
  \else
    \Gin@nat@width
  \fi
}
\makeatother

\IfFileExists{upquote.sty}{\usepackage{upquote}}{}
\definecolor{fgcolor}{rgb}{0.2, 0.2, 0.2}
\newcommand{\hlnumber}[1]{\textcolor[rgb]{0,0,0}{#1}}%
\newcommand{\hlfunctioncall}[1]{\textcolor[rgb]{0.501960784313725,0,0.329411764705882}{\textbf{#1}}}%
\newcommand{\hlstring}[1]{\textcolor[rgb]{0.6,0.6,1}{#1}}%
\newcommand{\hlkeyword}[1]{\textcolor[rgb]{0,0,0}{\textbf{#1}}}%
\newcommand{\hlargument}[1]{\textcolor[rgb]{0.690196078431373,0.250980392156863,0.0196078431372549}{#1}}%
\newcommand{\hlcomment}[1]{\textcolor[rgb]{0.180392156862745,0.6,0.341176470588235}{#1}}%
\newcommand{\hlroxygencomment}[1]{\textcolor[rgb]{0.43921568627451,0.47843137254902,0.701960784313725}{#1}}%
\newcommand{\hlformalargs}[1]{\textcolor[rgb]{0.690196078431373,0.250980392156863,0.0196078431372549}{#1}}%
\newcommand{\hleqformalargs}[1]{\textcolor[rgb]{0.690196078431373,0.250980392156863,0.0196078431372549}{#1}}%
\newcommand{\hlassignement}[1]{\textcolor[rgb]{0,0,0}{\textbf{#1}}}%
\newcommand{\hlpackage}[1]{\textcolor[rgb]{0.588235294117647,0.709803921568627,0.145098039215686}{#1}}%
\newcommand{\hlslot}[1]{\textit{#1}}%
\newcommand{\hlsymbol}[1]{\textcolor[rgb]{0,0,0}{#1}}%
\newcommand{\hlprompt}[1]{\textcolor[rgb]{0.2,0.2,0.2}{#1}}%

\usepackage{framed}
\makeatletter
\newenvironment{kframe}{%
 \def\at@end@of@kframe{}%
 \ifinner\ifhmode%
  \def\at@end@of@kframe{\end{minipage}}%
  \begin{minipage}{\columnwidth}%
 \fi\fi%
 \def\FrameCommand##1{\hskip\@totalleftmargin \hskip-\fboxsep
 \colorbox{shadecolor}{##1}\hskip-\fboxsep
     % There is no \\@totalrightmargin, so:
     \hskip-\linewidth \hskip-\@totalleftmargin \hskip\columnwidth}%
 \MakeFramed {\advance\hsize-\width
   \@totalleftmargin\z@ \linewidth\hsize
   \@setminipage}}%
 {\par\unskip\endMakeFramed%
 \at@end@of@kframe}
\makeatother

\definecolor{shadecolor}{rgb}{.97, .97, .97}
\definecolor{messagecolor}{rgb}{0, 0, 0}
\definecolor{warningcolor}{rgb}{1, 0, 1}
\definecolor{errorcolor}{rgb}{1, 0, 0}
\newenvironment{knitrout}{}{} % an empty environment to be redefined in TeX

\usepackage{alltt}
\usetheme{Stats}
\setbeamercovered{transparent}
\usepackage{color}
\usepackage{hyperref}
  \hypersetup{
  	colorlinks=true
		linkcolor=black
		}
\usepackage{url}
\usepackage{graphics}
\usepackage{tikz}
\usepackage{booktabs}





%%%%%%%%%%%%%%%%%%%%%%%%%%%%%%%% Title Slide %%%%%%%%%%%%%%%%%%%%%%%%%%
\title[]{Intro to Social Science Data Analysis \\[1cm] Week 12 Seminar: Multivariate Linear Regression \& Presenting Regression Results}
\author[]{
    \href{mailto:gandrud@yonsei.ac.kr}{Christopher Gandrud}
}
\date{\today}


\begin{document}

\frame{\titlepage}


%%%%%%%%%% Assignment 4
\section{Assignment 4}
\frame{
  \frametitle{Assignment 4}
  Due: Friday 30 November \\[0.5cm]
  {\Large{Research Design}} \\[0.25cm]
  With your partner plan your research by answering the following questions:
  \begin{enumerate}
    \item What \textbf{difference} do you want to explain?
    \item What is your \textbf{best guess} explanation (i.e. thesis statement)?
    \item Can you \textbf{test your hypothesis using data}? If so, what data do you need to collect and what tests could you use?
    \item What \textbf{rival explanations} are their?
    \item How could you use data to test whether your best guess or the rival explanations are better? Write this as an \textbf{equation}.
  \end{enumerate}
{\tiny{Questionnaire from: modified from Cheryl Schonhardt-Bailey}}
}

\begin{frame}[fragile]
	\frametitle{Load Data}
\begin{knitrout}
\definecolor{shadecolor}{rgb}{0.969, 0.969, 0.969}\color{fgcolor}\begin{kframe}
\begin{alltt}
\hlcomment{# Load openintro package}
\hlfunctioncall{library}(openintro)

\hlcomment{# Load Data}
\hlfunctioncall{data}(census)

\hlcomment{# Show variable names}
\hlfunctioncall{names}(census)
\end{alltt}
\begin{verbatim}
## [1] "censusYear"          "stateFIPScode"      
## [3] "totalFamilyIncome"   "age"                
## [5] "sex"                 "raceGeneral"        
## [7] "maritalStatus"       "totalPersonalIncome"
\end{verbatim}
\end{kframe}
\end{knitrout}

\end{frame}

\frame{
  \frametitle{Model}
  With a partner, hypothesize what the likely associations between the variables:
  \begin{itemize}
    \item \texttt{age},
    \item \texttt{sex},
    \item \texttt{raceGeneral},
    \item \texttt{maritalStatus},
  \end{itemize}
  with \texttt{totalPersonalIncome}.
}

\frame{
  \frametitle{Predict the Effect}
  Using all of the variables in the data set create a parsimonious, but comprehensive linear regression model to find a \textbf{point estimate} of the total presonal income of a white widowed women who is 32 years old.  \\[0.5cm]
  Write the linear regression equation and make the prediction.
}

\frame{
  \frametitle{Simulations}
  Simulate expected total family incomes, with associated uncertainty, for a range of ages.
}
\end{document}
